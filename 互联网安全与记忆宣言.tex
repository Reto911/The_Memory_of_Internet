\documentclass[a4paper,12pt]{ctexart}
\usepackage{zhnumber,syntonly}
\renewcommand{\thesection}{\zhnum{section}、}
%\syntaxonly
\setCJKfamilyfont{msyh}{微软雅黑}
\newcommand{\msyh}{\CJKfamily{msyh}}
\setCJKmainfont{华文楷体}
\title{互联网记忆与安全宣言}
\date{}
\begin{document}
	\maketitle
	\section*{引言}
	自诞生的那一刻起,互联网就被赋予了极其重要的使命。
	
	信息交流是互联网建立的初衷,且直到今天,这也仍是其最重要的职能。正因如此,在二十一世纪的今天,互联网得以被广泛运用到电子商务、休闲娱乐、企业经营、政府运作、基层治理等诸多应用情景中去。
	
	显然,人类的需求并不止步于此。低延时、易加密等特点在赋予其信息交流的职能的同时,也迎合着人们的另一样需求。于是,在信息交流的基础上,它又被赋予了新的职能:记忆寄存。
	
	这种现象发生的原因是显而易见的:现有的知识、记忆过多,而迄今为止,还没有哪种工具在记忆储存与分发的高效性、安全性、性价比上比互联网更具优势。今天的我们,只需动动手指,便可把数十年的影像记忆上传到互联网上加以储存,并在顷刻间与来自世界各地的人分享。这在上个世纪还只是个遥不可及的幻想,但随着传输速度与设备容量的飞速提升,互联网在记忆存储与分发上的应用日渐广泛。从早年遍地开花的个人主页到各大博客、播客网站,再到如今大红大紫的网络硬盘服务,互联网已经确切地担任起时代赋予的重任。
	
	如今,我们常把珍贵的照片录像上传到网络空间,以期它们能够长久保存。此外,我们还在互联网上畅所欲言、结交知音,并向他人展示自己的创作。这些由我们刻下的烙印理应得到妥善的保管。然而,事与愿违,在互联网的便利中安逸麻木的我们,难以意识到一场灾难正在悄然发生。
	
	毫无疑问,资本的介入极大地促进了互联网——尤其是记忆储存事业的发展。正是拥有丰厚资金与先进技术的企业的参与,带来了无数的技术革新。但在资本无底线地逐利的本性下,它们又开始阻碍、乃至束缚这些行业的发展,直至把互联网及其用户拖入无底深渊。毫不夸张地说,当下普通互联网用户的数据安全已经岌岌可危。此处的“安全”包括但不限于储存权、隐私权、处分权、删除权以及著作、知识权等。企业,尤其是大型、跨国企业,在数据产业上合纵连横,使我们逐渐失去了对自己的记忆的控制权。为了利益,它们不会、也不能作出任何承诺。它们完全可能私自使用、篡改、删除用户所交付的记忆,而不经过用户知情、同意,甚至强迫用户同意,而它们确实这么做了。事实已经向我们证明,在互联网时代,我们所寄存的记忆早已不属于我们自己,它们早已沦为企业在经济与政治等方面利益博弈的筹码。
	
	我们庆幸地看到前辈们创造了人类历史上最伟大的奇迹,却又遗憾地看到我们的敌人把这样的奇迹无情践踏。而这种践踏和蚕食正是我们所无法容忍的。Mozilla认为,“互联网是全球公共资源,必须保证开放性和可用性。”作为Mozilla的忠实拥护者,我们同样把这当做我们的信条。有鉴于此,我们认为我们有义务对一切曾有的、既有的以及将有的互联网的回忆与珍宝加以保护。为了实现这个目标,我们需要团结一切可以团结的力量,集中一切可以集中的资源,我们必须向全世界的有志之士传达我们的号召,而这就是我们发表这份《宣言》的原因。
	
	\newpage
	\section{现状}
	
	
	\section{原则}
	\section{使命}
	\section{承诺}
\end{document}